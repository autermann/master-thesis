%!TEX root = thesis.tex
\section{Introduction}
\section{Lake-Analyzer}
\section{Foundtations}
	\subsection{Limniography}
	\subsection{Web Processing Service}
	\subsection{Streaming}
		\begin{itemize}
			\item sequential instead of random access
			\item seeing outputs before everything is transmitted
			\item less resources needed to process data
			\item multimedia
			\begin{itemize}
				\item RTP, RTCP, RTSP, SIP
				\item on-demand
				\item live
				\item video
				\item audio
			\end{itemize}
			\item IPC
			\begin{itemize}
				\item Pipes
				\item sockets
			\end{itemize}
		\end{itemize}
		\subsubsection{WebSockets}
	\subsection{Matlab}
	\subsection{WPS4R}
	\subsection{Previous approaches}
	\begin{itemize}
		\item heavily format specific
		\begin{itemize}
			\item parsing of GML/etc in the WPS and translation to R structures
		\end{itemize}
		\item publishing results in a playlist that has to be checked constantly
		\item WPS splits inputs
		\item configuration as comments in R scripts
		\item focussing on scripts and not on functions
	\end{itemize}
\section{Matlab WPS}
\begin{itemize}
	\item matlab function <-> wps process
	\item not format specific
	\item no conversion of complex inputs/outputs
	\begin{itemize}
		\item single output formats
	\end{itemize}
	\item matlab program has to parse inputs
	\item easy to publish existing scripts and functions as WPS processes
	\item multi-tier implementation
	\begin{itemize}
		\item Matlab WPS
		\begin{itemize}
			\item Translates WPS Execute requests to Matlab client requests
			\item Translates Matlab client responses to WPS Execute responses
			\item configurationn with YAML file to create description and translate inputs/outputs
		\end{itemize}
		\item Matlab Client
		\begin{itemize}
			\item WebSocket client to access the Matlab server.
			\item offers simple request building API
		\end{itemize}
		\item Matlab Server
		\begin{itemize}
			\item WebSocket server that pools multiple Matlab Instances
			\item delegates requests to free instances
		\end{itemize}
		\item Matlab Instance
		\begin{itemize}
			\item a Java wrapper around a Matlab instance
		\end{itemize}
		\item Matlab
		\begin{itemize}
			\item A headless intance of the Matlab software
		\end{itemize}
	\end{itemize}
\end{itemize}
	\subsection{Configuration}
	\begin{itemize}
		\item Can not be used to offer any function as process
		\item would not conform to Mathworks license
		\item configuring of a single function as a process
		\item configuration YAML file
	\end{itemize}
	\includecode[Matlab]
		{matlab-add-function.m}
		{\label{lst:matlab:example:fun}Matlab example function.}
	\includecode[YAML,morekeywords={function,connection,identifier,version,inputs,outputs,type}]
		{matlab-add-process-configuration.yaml}
		{\label{lst:matlab:example:yaml}Matlab process configuration describing the function in Listing \ref{lst:matlab:example:fun}.}
	\includecode[XML]
		{matlab-add-process-description.xml}
		{\label{lst:matlab:example:desc}Process description generated from the configuration in Listing \ref{lst:matlab:example:yaml}.}

	\subsection{Type Mapping}
	\begin{table}[!htb]
		\sffamily\centering
		\caption{\label{tab:matlab:typemapping}Type Mapping between Matlab and WPS Data}
		\begin{tabular}{@{}llcc@{}}
			\toprule
			&
			& \multicolumn{2}{b}{Matlab Type}\\
			\cmidrule(l){3-4}
			\multicolumn{1}{@{}b}{}
			& \multicolumn{1}{b}{Data}
			& \multicolumn{1}{b}{For single inputs}
			& \multicolumn{1}{b@{}}{For multiple inputs}\\
			\cmidrule(rl){2-2}
			\cmidrule(rl){3-3}
			\cmidrule(l){4-4}
			\textbf{Complex}      & \textit{any} & String  & Cell \\\midrule
			\textbf{Bounding Box} & -            & -       & -    \\\midrule
			\textbf{Literal}      & xs:int       & Numeric & Array\\
							      & xs:boolean   & Numeric & Array\\
							      & xs:dateTime  & Numeric & Array\\
							      & xs:double    & Numeric & Array\\
							      & xs:float     & Numeric & Array\\
							      & xs:byte      & Numeric & Array\\
							      & xs:short     & Numeric & Array\\
							      & xs:int       & Numeric & Array\\
							      & xs:long      & Numeric & Array\\
							      & xs:string    & String  & Cell \\
							      & xs:anyURI    & String  & Cell \\
			\bottomrule
		\end{tabular}
	\end{table}
	\subsection{Pooling}
	\begin{itemize}
		\item matlab instances are pooled
		\item reduced starting time of instances
		\item limitation of instances
	\end{itemize}
	\subsection{License Issues}

		\begin{signedquote}{The MathWorks, Inc. Software License Agreement}
			4. LICENSE RESTRICTIONS.  The License is subject to the express restrictions
			set forth below. Licensee shall not, and shall not permit any Affiliate or any
			Third Party to:
				[...]
			    4.8. provide access (directly or indirectly) to the Programs via a web or
			    network Application, except as permitted in Article 8 of the Deployment
			    Addendum;
		\end{signedquote}

		\begin{signedquote}{The MathWorks, Inc. Software License Agreement - Deployment Addendum}
			8. WEB APPLICATIONS.  Licensee may not provide access to an entire Program
			or a substantial portion of a Program by means of a web interface.

			For the Network Concurrent User Activation Type.  Programs licensed under the
			Network Concurrent User Activation Type may be called via a web application,
			provided the web application does not provide access to the MATLAB command
			line, or any of the licensed Programs with code generation capabilities.  In
			addition, Licensed Users may not provide access to an entire Program or a
			substantial portion of a Program.  Such operation of an application via a web
			interface may be provided to an unlimited number of web browser clients, at no
			additional cost, for Licensee's own use for its Internal Operations, and for
			use by Third Parties.

			For the Network Named User and Standalone Named User Activation Types.
			Programs licensed under the Network Named User and Standalone Named User
			Activation Types may be called via a web application, provided the web
			application does not provide access to the MATLAB command line, or any of the
			licensed Programs with code generation capabilities, and such application is
			only accessed by designated Network Named User or Standalone Named User
			licensees of such Programs.

			Programs licensed under any other Activation Type may not be called via a web
			interface.
		\end{signedquote}
	\subsection{Implementation}
	\begin{figure}[!htb]
		\centering
		\includegraphics[width=.8\textwidth]{figures/sequence-diagramm-mwps.pdf}
		\caption{\label{fig:sd:mwps} Sequence diagram of the Matlab WPS.} %182x194
	\end{figure}
	\subsection{Lake-Analyzer WPS}
\section{Streaming WPS}
	\begin{itemize}
		\item full input/output streaming support
		\item easy wrapping of existing process into streming process
		\item easy wrapping of streaming process into standard WPS processes
		\item stateful streaming processes
		\item dependencies by referencing prevsious iterations or their outputs
		\item storage of every output (currently in memory)
		\begin{itemize}
			\item saving results in a database easily possible
		\end{itemize}
	\end{itemize}
	\subsection{Input Types}\label{sec:streaming:input-types}
	\subsubsection{Static Inputs}
		\begin{itemize}
			\item see Listing \ref{lst:streaming:input:static}
			\item inputs common to every streaming iteration
			\item supplied when starting the streaming process
			\item merged with streaming inputs and forwarded to delegate
			\item configuration parameters, etc.
		\end{itemize}
		\includecode[XML]{streaming-input-static.xml}{\label{lst:streaming:input:static}Example for a Streaming WPS static inputs.}
	\subsubsection{Streaming Inputs}
		\begin{itemize}
			\item see Listing \ref{lst:streaming:input:streaming}
			\item provided for a single streaming iteration
			\item features all WPS input data types
		\end{itemize}
		\includecode[XML]{streaming-input-streaming.xml}{\label{lst:streaming:input:streaming}Example for a Streaming WPS streaming inputs.}
	\subsubsection{Reference Inputs}
		\begin{itemize}
			\item see Listing \ref{lst:streaming:input:reference}
			\item references the output of a previous or upcoming streaming iteration as an input for this iteration
			\item used to model dependencies between iterations/features/etc.
		\end{itemize}
		\includecode[XML]{streaming-input-reference.xml}{\label{lst:streaming:input:reference}Example for a Streaming WPS reference input.}
	\subsubsection{Polling inputs}
		\begin{itemize}
			\item Not implemented inside the streaming WPS.
			\item what to do if multiple polling inputs are defined?
			\item how to define polling frequency?
			\item how to define notifications?
			\item better handled on client side (see Figure \ref{fig:sd:polling})
		\end{itemize}
		\begin{figure}[!htb]
			\centering
			\includegraphics[width=.7868\textwidth]{figures/sequence-diagramm-polling.pdf}
			% 179x274
			\caption{\label{fig:sd:polling} Sequence diagram of the Streaming WPS.}
		\end{figure}

	\subsection{Dependencies}

	\begin{itemize}
		\item Directed acyclic graph $D=(V,E)$ with mutliple not interconnected subgraphs, probably quite sparse graphs.
		\item vertices are tasks, edges are dependencies to other tasks
		\item graph has to be acyclic as a cylce would produce a cyclic dependency that can not be resolved/executed
		\item see Figure \ref{fig:graph:unordered}
		\item topological ordering provides execution order see Figure \ref{fig:graph:ordered}
		\begin{itemize}
			\item ``A topological ordering, $ord_D$, of a directed acyclic graph $D = (V, E)$ maps each vertex to a priority value such that $ord_{D}(x) < ord_{D}(y)$ holds for all edges $x \rightarrow y \in E$.'' \citep{pearce2007dynamic}
			\item execution order = nodes sorted by descending $ord_D$
		 	\item linear time \citep{cormen2001introduction}: $\mathcal O(|V|+|E|)$  $\mathcal O(|E|)$ may vary between $\mathcal O(|V|)$ and  $O(|V|^2)$
		 \end{itemize}
		\item in contrast to e.g. package managers: non static graph, nodes and edges are added
		\item Recreating the topological ordering for every insertion can be costly \cite{pearce2007dynamic}
		\item \cite{pearce2007dynamic} provides dynamic algoritm for creating and maintaining the topological order, efficient on sparse graphs and constant factor slower on dense graphs
		\item graph only used to check for cyclic dependencies
		\item implementation listener based -> easier parallelization of execution
		\item streaming iterations are considered as tasks and can declare dependencies to other streaming iterations either by declaring mapping an input to the output of another streaming iteration (see Section \ref{sec:streaming:input-types}) or by declaring a explicit dependency on another streaming iteration
	\end{itemize}
	\begin{figure}[!htb]
		\centering
		\includegraphics[width=.4474\textwidth]{figures/unordered-graph.pdf} % 98x92
		\caption{\label{fig:graph:unordered} Example for a dependency graph.}
	\end{figure}
	\begin{figure}[!htb]
		\centering
		\includegraphics[width=1\textwidth]{figures/ordered-graph.pdf} % 219x58
		\caption{\label{fig:graph:ordered} Topological ordering for of the dependency graph in Figure \ref{fig:graph:unordered}.}
	\end{figure}

	\subsection{Protocoll}
	\begin{itemize}
		\item Starting of Streaming Process by executing a WPS process
		\item providing inputs and receiving outputs over WebSockets
		\item SOAP based message formats
		\item Messages:
		\begin{itemize}
			\item InputMessage
			\begin{itemize}
				\item used to provide inputs to a streaming iteration to the process
			\end{itemize}
			\item OutputMessage
			\begin{itemize}
				\item used to send outputs of a streaming iteration to the client
			\end{itemize}
			\item OutputRequestMessage
			\begin{itemize}
				\item used to request the outputs of a process from the process
			\end{itemize}
			\item ErrorMessage
			\begin{itemize}
				\item used to transport errors to the clients
			\end{itemize}
			\item DescribeMessage
			\begin{itemize}
				\item used to request a description of a streaming process from the process
			\end{itemize}
			\item DescriptionMessage
			\begin{itemize}
				\item used to describe a process to the client
			\end{itemize}
			\item StopMessage
			\begin{itemize}
				\item used to stop the streaming process, to indicate no further inputs will become available
				\item used to to notify any listening clients that no more outputs become available
			\end{itemize}
		\end{itemize}
	\end{itemize}
	\subsection{Implementation}
	\begin{itemize}
		\item based on the 52°North WPS
		\item includeable module
		\item default implementation uses another WPS process as delegate
	\end{itemize}
	\begin{figure}[!htb]
		\centering
		\includegraphics[width=.7868\textwidth]{figures/sequence-diagramm-swps.pdf}
		% 179x274
		\caption{\label{fig:sd:swps} Sequence diagram of the Streaming WPS.}
	\end{figure}
	\subsection{Client Implementation}
	\begin{itemize}
		\item small JavaScript library
		\item abstracts the message generation and websocket interaction
		\item may be used to start generic delegation processes
	\end{itemize}
	\subsection{Streaming Lake-Analyzer WPS}
	\begin{itemize}
		\item simple application of the Streaming WPS and Matlab WPS
		\item LakeAnalyzer may need further adjustements to allow live analysis
		\item remove downsampling code
		\item operate on single point in time
		\item etc
	\end{itemize}
	\subsection{Limitations}
	\begin{itemize}
		\item No input/output conversion
		\item Only default format is requested from delegate
		\item process will not fail fast in under every condition
		\begin{itemize}
			\item inputs first are checked at execution time
		\end{itemize}
		\item receivers are only provided with upcoming
		\begin{itemize}
			\item no replay queue
		\end{itemize}
	\end{itemize}
\section{Future Work}
\section{Conclusion}
% !TEX root = ../thesis.tex

\chapter{Discussion, Conclusion \& Future Work}
\label{sec:conclusion}

This thesis had to deal with two independent sub problems: on the one hand, the domain expert friendly deployment of MATLAB models and algorithms as interoperable web services and on the other hand, the efficient processing of large data sets and the analysis of live or near real-time data.

The \ac{OGC} \acl{WPS} presents an ideal solution for web based processing of spatiotemporal data in an interoperable manner and allows the reuse of developed models and algorithms using a standardized interface. The introduced MATLAB WPS facilitates domain experts to expose domain specific models as WPS processes easily. This allows the participation of MATLAB software in the Model Web \citep{geller2008looking}, which promotes usage of models in processing chains and multi-model computations. The MATLAB WPS is not only of great importance for interoperable model development and sharing of research results, but thus also allows advanced processing chains that bridge the gap between different domains. By this, the MATLAB WPS supports the advance of research and can help to gain new insights by connecting domains, topics and themes that were previously considered independent.

Future improvements to the MATLAB WPS may include selective output generation so that only outputs requested by the client are generated. This can greatly reduce processing time and thus costs. Also the generation of different output formats has to be solved without introducing additional complexity. Both points may be solved by offering WPS process meta data to the MATLAB process (e.g. using a global variable). This should also include informations like the WPS's URL, etc. The \la shows that MATLAB functions that use a multitude of inputs and/or outputs quickly become hard to read and thus hard to manage. The optional encoding of inputs and outputs as structures could improve maintenance efforts for these kind of processes.

Streaming of spatiotemporal data offers a great solution to the problem of processing large data sets or live environmental -- and thus indefinite large -- data sets. Previous approaches are insufficient in regard to compatibility, efficiency and ultimately effectiveness. This is largely caused by the constraints and limitations imposed by the WPS standard. While being generic in terms of which processes can be encapsulated, it enforces strict constraints about how these processes are executed and how they can be described. Even though the asynchronous process execution offers a great leverage point for implementing streaming processes, the concept is too inflexible and lacks capabilities to transport intermediate outputs or even a reference to these. Processes are considered entities that can be executed and described, whereas process executions (or process instances) are not identifiable resources, but are defined as the inaccessible procedure of the \emph{Execute} operation. Apart from the fact that it is impossible to supply subsequent input parameters to a process, other operations that could be considered useful in various use cases are not designated. Even though status responses of asynchronously executed processes can have the status \emph{paused}, pausing/resuming or aborting of running processes can not be accomplished using the WPS interface and can not be implemented as long as process instances can not be identified. Furthermore, the WPS standard is fixed to HTTP and other transport layers that can be considered ideal for streaming, like WebSockets, are not envisaged.
As the WPS specification itself, the process description format is simple and thus is easily adoptable but eventually lacks expressiveness. Dependencies between inputs or dependencies between specific outputs and inputs can not be expressed and extensions, e.g. the differentiation between intermediate and final results, are not foreseen.

The approach taken in this thesis to implement streaming processing of geospatial data by breaking out of the WPS standard has to be critically evaluated, as it also breaks compatibility to existing client solutions. But as the above-mentioned limitations show, it is ultimately required to bypass the WPS specification in order to enable true streaming processing. Nevertheless, client solutions need to be adjusted to streaming inputs and outputs. This is even true for previous approaches that target WPS compatibility. Implementing the here described concept of streaming by sending and receiving messages using WebSockets does not require more intrusive changes than the comprehension of playlist files, the continuous polling for new outputs, the provision of data fragments as resources that can be referenced, or the maintenance of an accessible playlist file. By reusing WPS terminologies and technologies, not only large parts of existing client solutions can be reused, but many existing process implementations can be transformed to streaming processes. Due to its advanced dependency handling and the capability to handle multiple streaming inputs and outputs, the presented Streaming WPS is also applicable to more use cases.

Future research should evaluate the not yet finished and published WPS 2.0 specification \citep{ogc:wps2swg}. Even though the standardization process is not transparent to non-OGC members, the public available change requests indicate a rethinking of the status of process executions \citep[e.g. ][]{ogc:wps:cr109}. Despite increasing the compatibility to the WPS standard, future research should concentrate on an in-depth performance evaluation of the Streaming WPS. Additionally, the generic streaming process should be enhanced to be able to request non-default formats from delegates. Future development should also conceptualize a mechanism to declare which outputs should be outputted, which are only for internal use, or which are not required at all. The current approach of identical input and output definitions for referenced inputs should be enhanced to enable a conversion between different formats. Dependency resolution mechanisms should be enhanced so that true indefinite processing of live data becomes possible. The current approach indefinitely keeps references to previous streaming iterations with the result that at one point the dependency graph would not fit into the server's memory. An automatic eviction of old streaming iterations or the optional deactivation of dependency resolution should be examined. Web browser clients that connect only occasionally to a streaming process should also be able to request outputs that were generated prior to the output request message and not only upcoming outputs. For this, a (temporal restricted) replay queue could be developed.

The most important topic for future research is the development of appropriate client solutions that are able to supply inputs to a streaming process and are able to visualize their outputs. Besides that, similar extensions to OGC web services for data warehousing (e.g. \ac{WCS} or \ac{SOS}) need to be developed and existing approaches \citep[e.g. for the \ac{WFS}, see ][]{aydin2006streaming} need to be integrated.

This thesis showed on the example of \la how MATLAB analysis software can be exposed as interoperable WPS processes and how streaming can be used to process (indefinite) large data sets in a service-oriented environment, but forthcoming work has yet to demonstrate how the \la and its streaming variant can be integrated into existing work flows or environments, and which advanced models and processing chains can be developed using the MATLAB and Streaming WPS.
